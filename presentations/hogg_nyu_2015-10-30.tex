\documentclass[pdftex]{beamer}
%\input{vc}
% 1.77778 is the ratio of 16 to 9
\setlength{\paperheight}{2.5in} % going way small!
\setlength{\paperwidth}{1.77778\paperheight}
% 1.33333 is the ratio of 4 to 3
%\setlength{\paperheight}{3.0in} % way small!
%\setlength{\paperwidth}{1.33333\paperheight}
%
\setlength{\textwidth}{0.85\paperwidth}
\setlength{\textheight}{0.95\paperheight}
% import the next thing *after* the papersize
\usepackage{amssymb,amsmath,mathrsfs}
\usecolortheme{default}

% this one is debatable
\renewcommand{\emph}[1]{\textbf{#1}}

%%% color commands
\newcommand{\whiteonblack}{%
  \colorlet{fg}{white}
  \colorlet{bg}{black}
  \setbeamercolor{normal_text}{fg=white,bg=black}
  \setbeamercolor{background canvas}{fg=white,bg=black}
  \setbeamercolor{alerted_text}{fg=yellow}
  \setbeamercolor{example_text}{fg=white}
  \setbeamercolor{structure}{fg=white}
  \setbeamercolor{palette_quaternary}{fg=white}
}
\newcommand{\blackonwhite}{%
  \colorlet{fg}{black}
  \colorlet{bg}{white}
  \setbeamercolor{normal_text}{fg=black,bg=white}
  \setbeamercolor{background canvas}{fg=black,bg=white}
  \setbeamercolor{alerted_text}{fg=blue}
  \setbeamercolor{example_text}{fg=black}
  \setbeamercolor{structure}{fg=black}
  \setbeamercolor{palette_quaternary}{fg=black}
}
\xdefinecolor{pink}{rgb}{1.0,0.9,0.9}

%%% size and shape commands
\newlength{\figurewidth}
\setlength{\figurewidth}{\textwidth}
\newlength{\figureheight}
\setlength{\figureheight}{0.9\textheight}

%%% text commands
\newcommand{\project}[1]{\textsl{#1}}
  \newcommand{\tc}{\project{The~Cannon}}
  \newcommand{\an}{\project{Astrometry.net}}
  \newcommand{\euclid}{\project{Euclid}}
  \newcommand{\flickr}{\project{flickr}}
  \newcommand{\gaia}{\project{Gaia}}
  \newcommand{\galex}{\project{GALEX}}
  \newcommand{\kepler}{\project{Kepler}}
  \newcommand{\GALEX}{\galex}
  \newcommand{\hst}{\project{HST}}
  \newcommand{\hipparcos}{\project{Hipparcos}}
  \newcommand{\lsst}{\project{LSST}}
  \newcommand{\sdss}{\project{SDSS}}
  \newcommand{\sdssiii}{\project{SDSS-III}}
  \newcommand{\sdssiv}{\project{SDSS-IV}}
  \newcommand{\boss}{\project{BOSS}}
  \newcommand{\apogee}{\project{APOGEE}}
  \newcommand{\osss}{\project{OSSS}}
  \newcommand{\ska}{\project{SKA}}
  \newcommand{\vo}{\project{VO}}
  \newcommand{\rttd}{\project{Right Thing To Do}$^{\mbox{\scriptsize\sffamily{TM}}}$}
\newcommand{\foreign}[1]{\textit{#1}}
\newcommand{\latin}[1]{\foreign{#1}}
  \newcommand{\cf}{\latin{cf.}}
  \newcommand{\eg}{\latin{e.g.}}
  \newcommand{\etal}{\latin{et~al.}}
  \newcommand{\etc}{\latin{etc.}}
  \newcommand{\ie}{\latin{i.e.}}
  \newcommand{\vs}{\latin{vs.}}

%%% math-mode commands
\newcommand{\unit}[1]{\mathrm{#1}}
  \newcommand{\rad}{\unit{rad}}
  \newcommand{\s}{\unit{s}}
  \newcommand{\yr}{\unit{yr}}
  \newcommand{\km}{\unit{km}}
  \newcommand{\kmps}{\km\,\s^{-1}}
\newcommand{\mmatrix}[1]{\boldsymbol{#1}}
\newcommand{\tv}[1]{\boldsymbol{#1}}
\newcommand{\dd}{\mathrm{d}}
\newcommand{\given}{\,|\,}
\newcommand{\transpose}[1]{{#1}^{\mathsf{\!T}}}
\DeclareMathOperator*{\diag}{diag}
 % hogg standard colors

\title{Finding planets in very noisy data from stars}
\author[David W. Hogg (NYU)]{David W. Hogg
  \textsl{\footnotesize(NYU \& MPIA \& SCDA)}\\
  \texttt{@davidwhogg}\\[1ex]
  \textsl{\footnotesize
    in collaboration with:
    Dan~Foreman-Mackey~\textsl{(UW)},
    Bernhard~Sch\"olkopf~\textsl{(MPI-IS)},
    Dun~Wang~\textsl{(NYU)}}\\[1ex]
{Courant Applied Math Seminar / 2015 October 30}}

\newcommand{\conclusions}{%
\begin{frame}
  \frametitle{summary}
  \begin{itemize}
  \item exoplanet research is shaped by engineering and data analysis challenges
  \item the search for planets involves extracting tiny, sparse signals from (huge) data sets
    \begin{itemize}
    \item noise modeling and marginalization of systematics
    \end{itemize}
  \item Earth-like planets are plentiful in our Galaxy
    \begin{itemize}
    \item a percent (or more) of Sun-like stars host Earth-like planets
    \item the Solar System is not obviously typical in any way
    \item little is known about Jupiter-like planets
    \end{itemize}
  \end{itemize}
\end{frame}}

\begin{document}

\begin{frame}
  \titlepage
\end{frame}

\conclusions

\begin{frame}
  \frametitle{the NASA \kepler\ Mission}
  \begin{itemize}
  \item stared at 150,000 stars for 4 years (30-min cadence)
    \begin{itemize}
    \item 70,000 measurements per star
    \item all data completely public
    \end{itemize}
  \item looking for exoplanet transit signals
  \item found \emph{thousands} of exoplanets (candidates)
  \end{itemize}
\end{frame}

\begin{frame}
  \frametitle{the NASA \kepler\ Mission}
  ~\hfill
  \includegraphics<1>[height=\figureheight]{kepler/750603main_Ball_Kepler_A8468_275_lg_blog_main_horizontal.jpg}
  \includegraphics<2>[height=\figureheight]{kepler/Kepler_FOV_hiRes.jpg}
  \includegraphics<3>[height=\figureheight]{kepler/FirstLightLogInvertedPink_wslbld2400.jpg}
  \includegraphics<4>[height=0.9\figureheight]{1502.04715/figures-de-trended.pdf}
  \includegraphics<5>[height=0.9\figureheight]{1502.04715/figures-folded.pdf}
\end{frame}

\begin{frame}
  \frametitle{the NASA \kepler\ K2 Mission}
  \begin{itemize}
  \item failure of reaction wheels led to worse pointing
    \begin{itemize}
    \item forced to stare at fields such that Solar torque is near zero
    \end{itemize}
  \item staring at 12 fields for 80 days each
  \item all data immediately public
  \item finding exoplanets around lower-mass stars
    \begin{itemize}
    \item currently the longest list of K2 discoveries is from my group
    \item Foreman-Mackey \etal, arXiv:1502.04715
    \end{itemize}
  \end{itemize}
\end{frame}

\begin{frame}
  \frametitle{an Earth-like transit signal}
  \begin{itemize}
  \item requires good alignment (percent-level)
  \item Earth blocks $10^{-4}$ of the light from the Sun
  \item it does this for 13 hours out of every 365.25 days
  \item the Sun has stochastic variability with an amplitude larger than the signal
  \end{itemize}
\end{frame}

\begin{frame}
  \frametitle{noise models}
  \begin{itemize}
  \item precision astrophysics is all about noise
  \item stochastic variability of the Sun
    \begin{itemize}
    \item models of the Sun do not predict variability in detail
    \item we use data-driven models (non-parametrics)
    \end{itemize}
  \item spacecraft issues
    \begin{itemize}
    \item telescope pointing is not precise (especially for K2)
    \item we make use of shared information across all stars
    \end{itemize}
  \item both of these noise sources are \emph{larger than the signals we care about}
  \end{itemize}
\end{frame}

\begin{frame}
  \frametitle{raw K2 light curve {\footnotesize (Foreman-Mackey \etal, arXiv:1502.04715)}}
  ~\hfill
  \includegraphics<1>[trim=100 100 100 100, clip, height=\figureheight]{brownbag/brownbagp10.pdf}
\end{frame}

\begin{frame}
  \frametitle{fundamental approach}
  \begin{itemize}
  \item probabilistic inference
    \begin{itemize}
    \item write down $p(\mathrm{data}\given\mathrm{stuff})$
    \item search with likelihood ratio tests
    \item infer population properties with full Bayes and MCMC
    \end{itemize}
  \end{itemize}
\end{frame}

\begin{frame}
  \frametitle{causal structure}
  ~\hfill
  \includegraphics<1>[height=\figureheight]{../code/onecause.pdf}
  \includegraphics<2>[height=\figureheight]{../code/threecauses.pdf}
  \includegraphics<3>[height=\figureheight]{../code/morecauses.pdf}
\end{frame}

\begin{frame}
  \frametitle{causal idea}
  \begin{itemize}
  \item if a star's behavior can be predicted confidently by
    \emph{other stars} then that behavior must be being imprinted by
    the spacecraft (s/c)
  \item using these ideas to separate s/c-induced from
    intrinsic stellar variability (Wang \etal, in preparation)
  \end{itemize}
\end{frame}

\begin{frame}
  \frametitle{data-driven idea}
  \begin{itemize}
  \item The best predictors of s/c-induced effects
    empirical behaviors of \emph{other stars} (or pixels).
  \item The data \emph{are} the model.
  \end{itemize}
\end{frame}

\begin{frame}
  \frametitle{data-driven idea}
  \begin{itemize}
  \item Imagine that s/c pointing is the source of your
    dominant systematic.
  \item Imagine that you have thousands of light curves that are good to a part in
    $10^5$, every one of which depends differently on pointing.
  \item The light curves themselves contain the pointing information
    at very high fidelity.
  \item (Much higher fidelity than any direct s/c orientation measurements!)
  \end{itemize}
\end{frame}

\begin{frame}
  \frametitle{noise modeling {\footnotesize (Foreman-Mackey \etal, arXiv:1502.04715)}}
  ~\hfill
  \includegraphics<1>[trim=100 100 100 100, clip, height=\figureheight]{brownbag/brownbagp10.pdf}
  \includegraphics<2>[trim=100 100 100 100, clip, height=\figureheight]{brownbag/brownbagp11.pdf}
  \includegraphics<3>[trim=100 100 100 100, clip, height=\figureheight]{brownbag/brownbagp14.pdf}
  \includegraphics<4>[trim=100 100 100 100, clip, height=\figureheight]{brownbag/brownbagp15.pdf}
  \includegraphics<5>[trim=100 100 100 100, clip, height=\figureheight]{brownbag/brownbagp17.pdf}
\end{frame}

\begin{frame}
  \frametitle{exoplanet search}
  \begin{itemize}
  \item enormous set of hypothesis tests
    \begin{itemize}
    \item extremely fine grid in period and phase
    \item can't ``just'' use Fourier methods because of signal sparsity
    \end{itemize}
  \item very flexible noise model
    \begin{itemize}
    \item effectively hundreds to millions of parameters per lightcurve
    \item fully marginalize out all of these
    \end{itemize}
  \item clever applied mathematics
  \end{itemize}
\end{frame}

\begin{frame}
  \frametitle{reminder: $\chi^2$}
  \begin{itemize}
  \item $\displaystyle\chi^2\equiv\sum_n\frac{[y_n - \mu_n]^2}{\sigma_n^2}$
  \item ``fitting'' is (often) ``minimizing $\chi^2$''
  \item $\chi^2$ is related to the ln of a (Gaussian) likelihood.
  \end{itemize}
\end{frame}

\begin{frame}
  \frametitle{marginalizing out noise}
  \begin{itemize}
  \item $\displaystyle\chi^2\equiv\sum_n\frac{[y_n - \mu_n]^2}{\sigma_n^2}$
  \item $\displaystyle\chi^2\equiv\transpose{[y - \mu]}\cdot C^{-1}\cdot[y - \mu]$
  \item $\displaystyle C = \diag(\sigma^2)$
  \end{itemize}
\end{frame}

\begin{frame}
  \frametitle{marginalizing out noise}
  \begin{itemize}
  \item $\displaystyle\chi^2\equiv\transpose{[y - \mu]}\cdot C^{-1}\cdot[y - \mu]$
  \item $\displaystyle C = \diag(\sigma^2) + B\cdot\Lambda\cdot\transpose{B}$
    \begin{itemize}
    \item $B$ contains the $M=150$ (say) basis vectors
    \item either $\Lambda\rightarrow\infty$ (permitted!) or choose $\Lambda$ sensibly
    \item marginalizes out anything in the $B$ subspace
    \end{itemize}
  \end{itemize}
\end{frame}

\begin{frame}
  \frametitle{marginalizing out noise}
  \begin{itemize}
  \item $\displaystyle\chi^2\equiv\transpose{[y - \mu]}\cdot C^{-1}\cdot[y - \mu]$
  \item $\displaystyle C = \diag(\sigma^2) + B\cdot\Lambda\cdot\transpose{B} + K$
    \begin{itemize}
    \item $\displaystyle K_{nn'} = k(|t_{n} - t_{n'}|)$
    \item for example, $\displaystyle K_{nn'} = a\,\exp -\frac{1}{2}\,\frac{[t_{n} - t_{n'}]^2}{\tau^2}$
    \item marginalizes out smooth, nonlinear functions of time (a Gaussian Process)
    \end{itemize}
  \end{itemize}
\end{frame}

\begin{frame}
  \frametitle{marginalizing out noise}
  \begin{itemize}
  \item $\displaystyle\chi^2\equiv\transpose{[y - \mu]}\cdot C^{-1}\cdot[y - \mu]$
  \item These methods marginalize out (Gaussian) non-trivial noise models.
    \begin{itemize}
    \item equivalent to integrating over all noise draws consistent with the data
    \item there are more extreme (tractable) models that permit nonlinear functions of the $B$ vectors
    \end{itemize}
  \item They rely on deep assumptions.
    \begin{itemize}
    \item Gaussianity: break it and die.
    \item small skew and few outliers
    \end{itemize}
  \end{itemize}
\end{frame}

\begin{frame}
  \frametitle{exoplanet search {\footnotesize (Foreman-Mackey \etal, arXiv:1502.04715)}}
  ~\hfill
  \includegraphics<1>[height=0.9\figureheight]{1502.04715/figures-corr.pdf}
  \includegraphics<2>[height=0.9\figureheight]{1502.04715/figures-de-trended.pdf}
  \includegraphics<3>[height=0.9\figureheight]{1502.04715/figures-linear.pdf}
  \includegraphics<4>[height=0.9\figureheight]{1502.04715/figures-periodic.pdf}
  \includegraphics<5>[height=0.9\figureheight]{1502.04715/figures-de-trended.pdf}
  \includegraphics<6>[height=0.9\figureheight]{1502.04715/figures-folded.pdf}
  \includegraphics<7>[height=\figureheight]{1502.04715/figures-candidates.pdf}
\end{frame}

\begin{frame}
  \frametitle{false positives and false negatives}
  \begin{itemize}
  \item it is easy to figure out the \emph{false-negative} rate
    \begin{itemize}
    \item inject fake signals into the lightcurves, search
    \item complement is called the ``efficiency'' or ``completeness''
    \end{itemize}
  \item<2-> it is \emph{impossible} to figure out the \emph{false-positive} rate
    \begin{itemize}
    \item ground truth is known for no complete sample
    \item need a model for all sources (instrumental and astrophysical) of false positive signals
    \item the most interesting objects are nearly impossible to follow up
    \end{itemize}
  \end{itemize}
\end{frame}

\begin{frame}
  \frametitle{false positives and false negatives}
  \begin{itemize}
  \item \emph{secret fact:}
    it looks like many of the most interesting exoplanet candidates may be false positives
    \begin{itemize}
    \item long periods, small radii
    \end{itemize}
  \end{itemize}
\end{frame}

\begin{frame}
  \frametitle{search completeness {\footnotesize (Foreman-Mackey \etal, arXiv:1502.04715)}}
  ~\hfill
  \includegraphics<1>[height=\figureheight]{1502.04715/figures-completeness.pdf}
\end{frame}

\begin{frame}
  \frametitle{characterization: what can we measure?}
  \begin{itemize}
  \item easy: period, size ratio
  \item hard: size, temperature, composition
    \begin{itemize}
    \item We can't understand the exoplanets any better than we can understand the stars they orbit!
    \item renewed interest in precision stellar astrophysics
    \end{itemize}
  \end{itemize}
\end{frame}

\begin{frame}
  \frametitle{exoplanet populations {\footnotesize (Foreman-Mackey \etal, arXiv:1406.3020)}}
  ~\hfill
  \includegraphics<1>[height=\figureheight]{1406.3020/results-period.pdf}
  \includegraphics<2>[height=\figureheight]{1406.3020/results-radius.pdf}
  \includegraphics<3>[height=\figureheight]{1406.3020/results-rate.pdf}
  \includegraphics<4>[width=0.8\textwidth]{1406.3020/figures-comparison.pdf}
\end{frame}

\conclusions

\end{document}

%%% OLD STUFF

\begin{frame}
  \frametitle{reminder: $\chi^2$}
  \begin{eqnarray}
    \chi^2 &\equiv& \sum_i \frac{[y_i - m_i]^2}{\sigma_i^2}
    \nonumber \\
    \chi^2 &=& \transpose{[y - m]}\cdot V^{-1}\cdot [y - m]
    \nonumber \\
    V &=& C
    \nonumber \\
    C_{ij} &\equiv& \sigma_i^2\,\delta_{ij}
    \nonumber \\
    -2\,\ln L &=& \transpose{[y - m]}\cdot V^{-1}\cdot [y - m] + \ln\det V
    \nonumber
  \end{eqnarray}
\end{frame}

\begin{frame}
  \frametitle{reminder: $\chi^2$}
  \begin{itemize}
  \item $\transpose{[y - m]}\cdot V^{-1}\cdot [y - m] + \ln\det V$
  \item $y$ is your data, $m$ is your model (of an exoplanet signal!).
  \item $V$ encodes your beliefs about your (Gaussian) noise.
  \end{itemize}
\end{frame}

\begin{frame}
  \frametitle{Generalizing (Gaussian) noise}
  \begin{itemize}
  \item If there are multiple sources of (Gaussian) noise, their
    covariances add.
  \item Adding new components to the noise variance tensor $V$ is
    equivalent to fitting for other noise processes and marginalizing
    them out.
  \item This is literally as trivial as it sounds.
  \end{itemize}
\end{frame}

\begin{frame}
  \frametitle{Generalizing (Gaussian) noise}
  \begin{eqnarray}
    -2\,\ln L &=& \transpose{[y - m]}\cdot V^{-1}\cdot [y - m] + \ln\det V
    \nonumber \\
    V &=& C + Q
    \nonumber \\
    C_{ij} &\equiv& \sigma_i^2\,\delta_{ij}
    \nonumber \\
    Q &=& \mbox{variance tensor of additional noise}
    \nonumber
  \end{eqnarray}
\end{frame}
\begin{frame}
  \frametitle{Systematic noise model}
  \begin{itemize}
  \item Imagine that we believe that the spacecraft systematics
    (noise) can be modeled as a sum of $M$ basis vectors.
  \item If we can put a Gaussian prior on the linear amplitudes, we
    can fully represent this noise by adding in a rank-$M$ covariance
    matrix.
  \item This is equivalent to fitting and marginalizing!
  \end{itemize}
\end{frame}

\begin{frame}
  \frametitle{Systematic noise model}
  \begin{eqnarray}
    -2\,\ln L &=& \transpose{[y - m]}\cdot V^{-1}\cdot [y - m] + \ln\det V
    \nonumber \\
    V &=& C + B\cdot\Lambda\cdot\transpose{B}
    \nonumber \\
    B &=& \mbox{block of $M$ basis vectors}
    \nonumber \\
    \Lambda &=& \mbox{$M\times M$ prior variance; can $\rightarrow\infty$}
    \nonumber
  \end{eqnarray}
\end{frame}

\begin{frame}
  \frametitle{Systematic noise model}
  \begin{itemize}
  \item $y$ is your data, $m$ is your model (of an exoplanet signal!).
  \item $V + B\cdot\Lambda\cdot\transpose{B}$ encodes your beliefs
    about your (Gaussian) noise, which now has a systematics
    component.
  \end{itemize}
\end{frame}

\begin{frame}
  \frametitle{Systematic noise {\footnotesize (Foreman-Mackey \etal, 1502.04715)}}
  ~\hfill
  \includegraphics<1>[trim=100 100 100 100, clip, height=0.8\figureheight]{brownbag/brownbagp10.pdf}
  \includegraphics<2>[trim=100 100 100 100, clip, height=0.8\figureheight]{brownbag/brownbagp14.pdf}
  \includegraphics<3>[trim=100 100 100 100, clip, height=0.8\figureheight]{brownbag/brownbagp15.pdf}
  \includegraphics<4>[trim=100 100 100 100, clip, height=0.8\figureheight]{brownbag/brownbagp17.pdf}
\end{frame}

\begin{frame}
  \frametitle{Systematic noise model}
  \begin{itemize}
  \item How do we set $\Lambda$?
  \item Learn it:  Maximum likelihood or Bayes.
  \item (After all, we have a likelihood function!)
  \end{itemize}
\end{frame}

\begin{frame}
  \frametitle{Implementation details}
  \begin{itemize}
  \item For expressions like $C + B\cdot\Lambda\cdot\transpose{B}$
    there are matrix inversion and determinant lemmas.
  \item Never use \code{inv()}. Always use \code{solve()}.
  \item Never use \code{det()}. Always use \code{slogdet()}.
  \end{itemize}
\end{frame}

\begin{frame}
  \frametitle{Systematic noise model}
  \begin{itemize}
  \item Don't ``fit and subtract'' your systematics!
  \item That isn't conservative.
  \item Signals get distorted {\footnotesize (Foreman-Mackey \etal, 1502.04715)}
  \item You need to marginalize out the noise.
  \item Using $C + B\cdot\Lambda\cdot\transpose{B}$ in $\ln L$ is
    equivalent to fitting the noise simultaneously with your model $m$
    and then marginalizing it out!
  \end{itemize}
\end{frame}

\begin{frame}
  \frametitle{Enormous models}
  \begin{itemize}
  \item Don't be afraid of enormous models.
  \item We sometimes use of order $10^9$ nuisance parameters for a
    single \kepler\ lightcurve (70,000 points) {\footnotesize (Wang
      \etal, in prep)}.
  \item In general, for nuisances, you want to use very flexible
    models but to learn the ``priors''\footnote{They are not really
      priors if you learn them!}.
  \end{itemize}
\end{frame}

\begin{frame}
  \frametitle{Gaussians are miraculous}
  \begin{itemize}
  \item ~[that is all]
  \end{itemize}
\end{frame}

\begin{frame}
  \frametitle{Stellar variability noise model}
  \begin{itemize}
  \item Imagine that we believe that stars vary stochastically but
    continuously (\ie, with time correlations).
  \item For any model of the amplitude and covariance of the
    variations, we can construct a covariance matrix (to add in).
  \item This is equivalent to fitting and marginalizing!
  \end{itemize}
\end{frame}

\begin{frame}
  \frametitle{Stellar variability noise model}
  \begin{eqnarray}
    -2\,\ln L &=& \transpose{[y - m]}\cdot V^{-1}\cdot [y - m] + \ln\det V
    \nonumber \\
    V &=& C + K
    \nonumber \\
    K_{ij} &=& K(|t_i - t_j|; \theta)
    \nonumber
  \end{eqnarray}
\end{frame}

\begin{frame}
  \frametitle{Gaussian processes}
  ~\hfill
  \includegraphics<1>[height=0.8\figureheight]{george/ind-results.png}
  \includegraphics<2>[height=0.8\figureheight]{george/gp-results.png}
\end{frame}

\begin{frame}
  \frametitle{Implementation details}
  \begin{itemize}
  \item There are many applied-math tricks for making things much
    faster than $N^{2.6}$.
  \item Use \project{george} {\footnotesize (Foreman-Mackey \etal)}.
  \item Marginalizing out the kernel hyperparameters $\theta$ is (in
    general) very expensive.
  \end{itemize}
\end{frame}

\begin{frame}
  \frametitle{Systematic noise model}
  \begin{itemize}
  \item Again, don't ``fit and subtract'' the stellar variability!
  \item Again, using $C + K$ in $\ln L$ is equivalent to fitting the
    stochastic variability simultaneously and then marginalizing it
    out!
  \end{itemize}
\end{frame}

\begin{frame}
  \frametitle{Kernel learning}
  \begin{itemize}
  \item Most uses of Gaussian Processes make use of kernel matrices
    with made-up kernels.
  \item \eg, $K(|t_i-t_j|;\theta) = a^2 \exp -\frac{|t_i-t_j|^2}{2\,\tau^2}$
  \item There are enormous families of kernels constructable from
    functions in the Fourier domain (eg, mixtures of Gaussians).
  \item With the data we have, we should be learning the kernels, not
    adopting convenient ones.
  \end{itemize}
\end{frame}

\begin{frame}
  \frametitle{Kernel learning {\footnotesize (Wilson \& Adams, 1302.4245)}}
  ~\hfill
  \includegraphics<1>[height=0.8\figureheight]{george/example_run_letters.jpg}
\end{frame}

\conclusions

\end{document}
