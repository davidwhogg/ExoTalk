\documentclass[pdftex]{beamer}
%\input{vc}
% 1.77778 is the ratio of 16 to 9
\setlength{\paperheight}{3.5in}
\setlength{\paperwidth}{1.77778\paperheight}
% 1.33333 is the ratio of 4 to 3
%\setlength{\paperheight}{4.0in}
%\setlength{\paperwidth}{1.33333\paperheight}
\setlength{\textwidth}{0.85\paperwidth}
% import the next thing *after* the papersize
\usepackage{amssymb,amsmath,mathrsfs}
\usecolortheme{default}

% this one is debatable
\renewcommand{\emph}[1]{\textbf{#1}}

%%% color commands
\newcommand{\whiteonblack}{%
  \colorlet{fg}{white}
  \colorlet{bg}{black}
  \setbeamercolor{normal_text}{fg=white,bg=black}
  \setbeamercolor{background canvas}{fg=white,bg=black}
  \setbeamercolor{alerted_text}{fg=yellow}
  \setbeamercolor{example_text}{fg=white}
  \setbeamercolor{structure}{fg=white}
  \setbeamercolor{palette_quaternary}{fg=white}
}
\newcommand{\blackonwhite}{%
  \colorlet{fg}{black}
  \colorlet{bg}{white}
  \setbeamercolor{normal_text}{fg=black,bg=white}
  \setbeamercolor{background canvas}{fg=black,bg=white}
  \setbeamercolor{alerted_text}{fg=blue}
  \setbeamercolor{example_text}{fg=black}
  \setbeamercolor{structure}{fg=black}
  \setbeamercolor{palette_quaternary}{fg=black}
}
\xdefinecolor{pink}{rgb}{1.0,0.9,0.9}

%%% size and shape commands
\newlength{\figurewidth}
\setlength{\figurewidth}{\textwidth}
\newlength{\figureheight}
\setlength{\figureheight}{0.9\textheight}

%%% text commands
\newcommand{\project}[1]{\textsl{#1}}
  \newcommand{\tc}{\project{The~Cannon}}
  \newcommand{\an}{\project{Astrometry.net}}
  \newcommand{\euclid}{\project{Euclid}}
  \newcommand{\flickr}{\project{flickr}}
  \newcommand{\gaia}{\project{Gaia}}
  \newcommand{\galex}{\project{GALEX}}
  \newcommand{\kepler}{\project{Kepler}}
  \newcommand{\GALEX}{\galex}
  \newcommand{\hst}{\project{HST}}
  \newcommand{\hipparcos}{\project{Hipparcos}}
  \newcommand{\lsst}{\project{LSST}}
  \newcommand{\sdss}{\project{SDSS}}
  \newcommand{\sdssiii}{\project{SDSS-III}}
  \newcommand{\sdssiv}{\project{SDSS-IV}}
  \newcommand{\boss}{\project{BOSS}}
  \newcommand{\apogee}{\project{APOGEE}}
  \newcommand{\osss}{\project{OSSS}}
  \newcommand{\ska}{\project{SKA}}
  \newcommand{\vo}{\project{VO}}
  \newcommand{\rttd}{\project{Right Thing To Do}$^{\mbox{\scriptsize\sffamily{TM}}}$}
\newcommand{\foreign}[1]{\textit{#1}}
\newcommand{\latin}[1]{\foreign{#1}}
  \newcommand{\cf}{\latin{cf.}}
  \newcommand{\eg}{\latin{e.g.}}
  \newcommand{\etal}{\latin{et~al.}}
  \newcommand{\etc}{\latin{etc.}}
  \newcommand{\ie}{\latin{i.e.}}
  \newcommand{\vs}{\latin{vs.}}

%%% math-mode commands
\newcommand{\unit}[1]{\mathrm{#1}}
  \newcommand{\rad}{\unit{rad}}
  \newcommand{\s}{\unit{s}}
  \newcommand{\yr}{\unit{yr}}
  \newcommand{\km}{\unit{km}}
  \newcommand{\kmps}{\km\,\s^{-1}}
\newcommand{\mmatrix}[1]{\boldsymbol{#1}}
\newcommand{\tv}[1]{\boldsymbol{#1}}
\newcommand{\dd}{\mathrm{d}}
\newcommand{\given}{\,|\,}
\newcommand{\transpose}[1]{{#1}^{\mathsf{\!T}}}
\DeclareMathOperator*{\diag}{diag}
 % hogg standard colors

\title{Extra-solar planets: Search, characterization, and population inferences}
\author[David W. Hogg (NYU)]{David W. Hogg \\
  \textsl{\small Center for Cosmology and Particle Physics,
                 New York University} \\
  \textsl{\small Center for Data Science,
                 New York University} \\
  \textsl{\small Max-Planck-Insitut f\"ur Astronomie, Heidelberg}}
\date{2015 February 27}

\newcommand{\conclusions}{%
\begin{frame}
  \frametitle{conclusions}
  \begin{itemize}
  \item exoplanet research is shaped by engineering and data analysis challenges
    \begin{itemize}
    \item I am going to focus on these issues
    \end{itemize}
  \item \emph{search} involves extracting tiny, sparse signals from (huge) data sets
  \item \emph{characterization} leaves us with noisy individual-planet measurements
  \item \emph{population inferences} require expensive noise propagation
    \begin{itemize}
    \item hierarchical probabilistic inference
    \end{itemize}
  \item Earth-like planets are plentiful in our Galaxy
    \begin{itemize}
    \item a percent (or more) of Sun-like stars host Earth-like planets
    \item however, few solar systems ``look like'' ours
    \item little is known about Jupiter-like planets
    \end{itemize}
  \end{itemize}
\end{frame}}

\begin{document}

\conclusions

\begin{frame}
  \titlepage
  in collaboration with:\\
  \emph{Dan~Foreman-Mackey}~(NYU),
  Ben~Montet~(Caltech),
  Tim~Morton~(Princeton),
  Bernhard~Sch\"olkopf~(MPI-IS),
  Dun~Wang~(NYU)
\end{frame}

\begin{frame}
  \frametitle{the NASA \kepler\ Mission}
  \begin{itemize}
  \item stared at 150,000 stars for 4 years (30-min cadence)
    \begin{itemize}
    \item 70,000 measurements per star
    \item all data completely public
    \end{itemize}
  \item looking for exoplanet transit signals
  \item found \emph{thousands} of exoplanets (candidates)
  \end{itemize}
\end{frame}

\begin{frame}
  \frametitle{the NASA \kepler\ Mission}
  ~\hfill
  \includegraphics<1>[height=\figureheight]{kepler/750603main_Ball_Kepler_A8468_275_lg_blog_main_horizontal.jpg}
  \includegraphics<2>[height=\figureheight]{kepler/Kepler_FOV_hiRes.jpg}
  \includegraphics<3>[height=\figureheight]{kepler/FirstLightLogInvertedPink_wslbld2400.jpg}
  \includegraphics<4>[height=0.9\figureheight]{1502.04715/figures-de-trended.pdf}
  \includegraphics<5>[height=0.9\figureheight]{1502.04715/figures-folded.pdf}
\end{frame}

\begin{frame}
  \frametitle{the NASA \kepler\ K2 Mission}
  \begin{itemize}
  \item failure of reaction wheels led to worse pointing
    \begin{itemize}
    \item forced to stare at fields such that Solar torque is near zero
    \end{itemize}
  \item staring at 12 fields for 80 days each
  \item all data immediately public
  \item finding exoplanets around lower-mass stars
    \begin{itemize}
    \item currently the longest list of K2 discoveries is from my group
    \item Foreman-Mackey \etal, arXiv:1502.04715
    \end{itemize}
  \end{itemize}
\end{frame}

\begin{frame}
  \frametitle{an Earth-like transit signal}
  \begin{itemize}
  \item requires good alignment (percent-level)
  \item Earth blocks $10^{-4}$ of the light from the Sun
  \item it does this for 13 hours out of every 365.25 days
  \item the Sun has stochastic variability with an amplitude larger than the signal
  \end{itemize}
\end{frame}

\begin{frame}
  \frametitle{noise models}
  \begin{itemize}
  \item precision astrophysics is all about noise
  \item stochastic variability of the Sun
    \begin{itemize}
    \item models of the Sun do not predict variability in detail
    \item we use data-driven models (non-parametrics)
    \item auto-regressive models, Gaussian Processes
    \end{itemize}
  \item spacecraft issues
    \begin{itemize}
    \item telescope pointing is not precise (especially for K2)
    \item point-spread function and flat-field not known to sufficient precision
    \item temperature changes
    \item we make use of shared information across all stars
    \end{itemize}
  \item all of these noise sources are larger than the signals we care about
  \end{itemize}
\end{frame}

\begin{frame}
  \frametitle{causal ideas}
\end{frame}

\begin{frame}
  \frametitle{noise modeling {\footnotesize (Foreman-Mackey \etal, arXiv:1502.04715)}}
  ~\hfill
  \includegraphics<1>[trim=100 100 100 100, clip, height=\figureheight]{brownbag/brownbagp10.pdf}
  \includegraphics<2>[trim=100 100 100 100, clip, height=\figureheight]{brownbag/brownbagp11.pdf}
  \includegraphics<3>[trim=100 100 100 100, clip, height=\figureheight]{brownbag/brownbagp14.pdf}
  \includegraphics<4>[trim=100 100 100 100, clip, height=\figureheight]{brownbag/brownbagp15.pdf}
  \includegraphics<5>[trim=100 100 100 100, clip, height=\figureheight]{brownbag/brownbagp17.pdf}
\end{frame}

\begin{frame}
  \frametitle{exoplanet search}
  \begin{itemize}
  \item enormous set of hypothesis tests
    \begin{itemize}
    \item extremely fine grid in period and phase
    \item can't ``just'' use Fourier methods because of signal sparsity
    \end{itemize}
  \item clever applied mathematics
    \begin{itemize}
    \item very fast linear algebra for Gaussian Processes
    \item Ambikasaran \etal, arXiv:1403.6015
    \item approximations that permit re-use of repeated calculation
    \item exploit embarassing parallelization
    \end{itemize}
  \end{itemize}
\end{frame}

\begin{frame}
  \frametitle{exoplanet search {\footnotesize (Foreman-Mackey \etal, arXiv:1502.04715)}}
  ~\hfill
  \includegraphics<1>[height=0.9\figureheight]{1502.04715/figures-corr.pdf}
  \includegraphics<2>[height=0.9\figureheight]{1502.04715/figures-de-trended.pdf}
  \includegraphics<3>[height=0.9\figureheight]{1502.04715/figures-linear.pdf}
  \includegraphics<4>[height=0.9\figureheight]{1502.04715/figures-periodic.pdf}
  \includegraphics<5>[height=0.9\figureheight]{1502.04715/figures-de-trended.pdf}
  \includegraphics<6>[height=0.9\figureheight]{1502.04715/figures-folded.pdf}
  \includegraphics<7>[height=\figureheight]{1502.04715/figures-candidates.pdf}
\end{frame}

\begin{frame}
  \frametitle{false positives and false negatives}
  \begin{itemize}
  \item it is easy to figure out the false-negative rate
    \begin{itemize}
    \item inject fake signals into the lightcurves, search
    \item inverse is called the ``efficiency'' or ``completeness''
    \end{itemize}
  \item<2> it is \emph{impossible} to figure out the false-positive rate
    \begin{itemize}
    \item ground truth is known for no complete sample
    \item the most interesting objects are nearly impossible to follow up
    \end{itemize}
  \item<2> \emph{secret fact:}
    it looks like many of the most interesting exoplanet candidates may be false positives
  \end{itemize}
\end{frame}

\begin{frame}
  \frametitle{search completeness {\footnotesize (Foreman-Mackey \etal, arXiv:1502.04715)}}
  ~\hfill
  \includegraphics<1>[height=\figureheight]{1502.04715/figures-completeness.pdf}
\end{frame}

\begin{frame}
  \frametitle{characterization: what can we measure?}
  \begin{itemize}
  \item foo
    \begin{itemize}
    \item bar
    \item bar
    \end{itemize}
  \item foo
    \begin{itemize}
    \item bar
    \item bar
    \end{itemize}
  \item foo
  \end{itemize}
\end{frame}

\begin{frame}
  \frametitle{how do you use noisy measurements?}
  \begin{itemize}
  \item foo
    \begin{itemize}
    \item bar
    \item bar
    \end{itemize}
  \item foo
    \begin{itemize}
    \item bar
    \item bar
    \end{itemize}
  \item foo
  \end{itemize}
\end{frame}

\begin{frame}
  \frametitle{hierarchical probabilistic modeling}
  \begin{itemize}
  \item foo
    \begin{itemize}
    \item bar
    \item bar
    \end{itemize}
  \item foo
    \begin{itemize}
    \item bar
    \item bar
    \end{itemize}
  \item foo
  \end{itemize}
\end{frame}

\begin{frame}
  \frametitle{whatever}
  \begin{itemize}
  \item foo
    \begin{itemize}
    \item bar
    \item bar
    \end{itemize}
  \item foo
    \begin{itemize}
    \item bar
    \item bar
    \end{itemize}
  \item foo
  \end{itemize}
\end{frame}

\begin{frame}
  \frametitle{Exoplanet populations {\footnotesize (Foreman-Mackey \etal, arXiv:1406.3020)}}
  ~\hfill
  \includegraphics<1>[height=\figureheight]{1406.3020/results-results.pdf}
  \includegraphics<2>[height=\figureheight]{1406.3020/results-period.pdf}
  \includegraphics<3>[height=\figureheight]{1406.3020/results-radius.pdf}
  \includegraphics<4>[height=\figureheight]{1406.3020/results-linear-radius.pdf}
  \includegraphics<5>[height=\figureheight]{1406.3020/results-rate.pdf}
  \includegraphics<6>[width=0.8\textwidth]{1406.3020/figures-comparison.pdf}
\end{frame}

\conclusions

\end{document}
