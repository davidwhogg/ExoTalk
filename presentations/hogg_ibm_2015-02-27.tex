\documentclass[pdftex]{beamer}
%\input{vc}
% 1.77778 is the ratio of 16 to 9
\setlength{\paperheight}{3.5in}
\setlength{\paperwidth}{1.77778\paperheight}
% 1.33333 is the ratio of 4 to 3
%\setlength{\paperheight}{4.0in}
%\setlength{\paperwidth}{1.33333\paperheight}
\setlength{\textwidth}{0.85\paperwidth}
% import the next thing *after* the papersize
\usepackage{amssymb,amsmath,mathrsfs}
\usecolortheme{default}

% this one is debatable
\renewcommand{\emph}[1]{\textbf{#1}}

%%% color commands
\newcommand{\whiteonblack}{%
  \colorlet{fg}{white}
  \colorlet{bg}{black}
  \setbeamercolor{normal_text}{fg=white,bg=black}
  \setbeamercolor{background canvas}{fg=white,bg=black}
  \setbeamercolor{alerted_text}{fg=yellow}
  \setbeamercolor{example_text}{fg=white}
  \setbeamercolor{structure}{fg=white}
  \setbeamercolor{palette_quaternary}{fg=white}
}
\newcommand{\blackonwhite}{%
  \colorlet{fg}{black}
  \colorlet{bg}{white}
  \setbeamercolor{normal_text}{fg=black,bg=white}
  \setbeamercolor{background canvas}{fg=black,bg=white}
  \setbeamercolor{alerted_text}{fg=blue}
  \setbeamercolor{example_text}{fg=black}
  \setbeamercolor{structure}{fg=black}
  \setbeamercolor{palette_quaternary}{fg=black}
}
\xdefinecolor{pink}{rgb}{1.0,0.9,0.9}

%%% size and shape commands
\newlength{\figurewidth}
\setlength{\figurewidth}{\textwidth}
\newlength{\figureheight}
\setlength{\figureheight}{0.9\textheight}

%%% text commands
\newcommand{\project}[1]{\textsl{#1}}
  \newcommand{\tc}{\project{The~Cannon}}
  \newcommand{\an}{\project{Astrometry.net}}
  \newcommand{\euclid}{\project{Euclid}}
  \newcommand{\flickr}{\project{flickr}}
  \newcommand{\gaia}{\project{Gaia}}
  \newcommand{\galex}{\project{GALEX}}
  \newcommand{\kepler}{\project{Kepler}}
  \newcommand{\GALEX}{\galex}
  \newcommand{\hst}{\project{HST}}
  \newcommand{\hipparcos}{\project{Hipparcos}}
  \newcommand{\lsst}{\project{LSST}}
  \newcommand{\sdss}{\project{SDSS}}
  \newcommand{\sdssiii}{\project{SDSS-III}}
  \newcommand{\sdssiv}{\project{SDSS-IV}}
  \newcommand{\boss}{\project{BOSS}}
  \newcommand{\apogee}{\project{APOGEE}}
  \newcommand{\osss}{\project{OSSS}}
  \newcommand{\ska}{\project{SKA}}
  \newcommand{\vo}{\project{VO}}
  \newcommand{\rttd}{\project{Right Thing To Do}$^{\mbox{\scriptsize\sffamily{TM}}}$}
\newcommand{\foreign}[1]{\textit{#1}}
\newcommand{\latin}[1]{\foreign{#1}}
  \newcommand{\cf}{\latin{cf.}}
  \newcommand{\eg}{\latin{e.g.}}
  \newcommand{\etal}{\latin{et~al.}}
  \newcommand{\etc}{\latin{etc.}}
  \newcommand{\ie}{\latin{i.e.}}
  \newcommand{\vs}{\latin{vs.}}

%%% math-mode commands
\newcommand{\unit}[1]{\mathrm{#1}}
  \newcommand{\rad}{\unit{rad}}
  \newcommand{\s}{\unit{s}}
  \newcommand{\yr}{\unit{yr}}
  \newcommand{\km}{\unit{km}}
  \newcommand{\kmps}{\km\,\s^{-1}}
\newcommand{\mmatrix}[1]{\boldsymbol{#1}}
\newcommand{\tv}[1]{\boldsymbol{#1}}
\newcommand{\dd}{\mathrm{d}}
\newcommand{\given}{\,|\,}
\newcommand{\transpose}[1]{{#1}^{\mathsf{\!T}}}
\DeclareMathOperator*{\diag}{diag}
 % hogg standard colors

\title{Extra-solar planets: Search, characterization, and population inferences}
\author[David W. Hogg (NYU)]{David W. Hogg \\
  \textsl{\small Center for Cosmology and Particle Physics,
                 New York University} \\
  \textsl{\small Center for Data Science,
                 New York University} \\
  \textsl{\small Max-Planck-Insitut f\"ur Astronomie, Heidelberg}}
\date{2015 February 27}

\newcommand{\conclusions}{%
\begin{frame}
  \frametitle{conclusions}
  \begin{itemize}
  \item search
  \item characterization
  \item populations
  \end{itemize}
\end{frame}}

\begin{document}

\conclusions

\begin{frame}
  \titlepage
  in collaboration with:\\
  \emph{Dan~Foreman-Mackey}~(NYU),
  Ben~Montet~(Caltech),
  Tim~Morton~(Princeton),
  Bernhard~Sch\"olkopf~(MPI-IS),
  Dun~Wang~(NYU)
\end{frame}

\begin{frame}
  \frametitle{the NASA \kepler\ Mission}
  ~\hfill
  \includegraphics<1>[height=\figureheight]{kepler/750603main_Ball_Kepler_A8468_275_lg_blog_main_horizontal.jpg}
  \includegraphics<2>[height=\figureheight]{kepler/Kepler_FOV_hiRes.jpg}
  \includegraphics<3>[height=\figureheight]{kepler/FirstLightLogInvertedPink_wslbld2400.jpg}
\end{frame}

\begin{frame}
  \frametitle{the paradox of precision astrophysics}
  \begin{itemize}
  \item models are incredibly \emph{explanatory}
    \begin{itemize}
    \item $\Lambda$CDM
    \item stellar spectroscopy
    \item helioseismology
    \end{itemize}
  \item and yet...
  \item<2-> models are \emph{wrong} (ruled out) in detail
    \begin{itemize}
    \item $\chi^2 \gg \nu$
    \item ``The $\chi^2$ statistic is a measure of the size of your data!''
    \end{itemize}
  \item<2-> missing physics, approximation, computation, \emph{gastrophysics}
  \end{itemize}
\end{frame}

\begin{frame}
  \frametitle{physics-driven models}
  \begin{itemize}
  \item put in everything you know
    \begin{itemize}
    \item gravity, atomic and molecular transitions, radiation
    \end{itemize}
  \item make approximations to make things computable
    \begin{itemize}
    \item subgrid models, mixing length, etc
    \end{itemize}
  \end{itemize}
\end{frame}

\begin{frame}
  \frametitle{data-driven models (my personal usage)}
  \begin{itemize}
  \item make use of things you \emph{strongly believe}
    \begin{itemize}
    \item noise model \& instrument resolution
    \item causal structure (shared parameters)
    \end{itemize}
  \item capitalize on huge amounts of data
  \item exceedingly flexible model
  \item concept of train, validate, and test
  \item every situation will be \emph{bespoke}
  \end{itemize}
\end{frame}

\begin{frame}
  \frametitle{Exoplanet populations {\footnotesize (Foreman-Mackey \etal, arXiv:1406.3020)}}
  ~\hfill
  \includegraphics<1>[height=\figureheight]{1406.3020/results-results.pdf}
  \includegraphics<2>[height=\figureheight]{1406.3020/results-period.pdf}
  \includegraphics<3>[height=\figureheight]{1406.3020/results-radius.pdf}
  \includegraphics<4>[height=\figureheight]{1406.3020/results-linear-radius.pdf}
  \includegraphics<5>[height=\figureheight]{1406.3020/results-rate.pdf}
  \includegraphics<6>[width=0.8\textwidth]{1406.3020/figures-comparison.pdf}
\end{frame}

\conclusions

\end{document}
