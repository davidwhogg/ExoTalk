\documentclass[pdftex]{beamer}
%\input{vc}

% set paper size
% 1.77778 is the ratio of 16 to 9
\setlength{\paperheight}{2.5in} % going way small!
\setlength{\paperwidth}{1.77778\paperheight}
% 1.33333 is the ratio of 4 to 3
%\setlength{\paperheight}{3.0in} % way small!
%\setlength{\paperwidth}{1.33333\paperheight}

% set lengths given paper
\setlength{\textheight}{0.95\paperheight}
\setlength{\textwidth}{0.85\paperwidth}

% import the next thing *after* the lengths and sizes
\input{./hogg_presentation} % hogg standard colors

\title{Discovering Exoplanets with Data-Driven Models}
\author[David W. Hogg (NYU)]{David W. Hogg \textsl{(SCDA \& NYU \& MPIA)} \\[1ex]
                             {\small in collaboration with Daniel Foreman-Mackey (UW)\\
                                     Dun Wang (NYU), Bernhard Sch\"olkopf (MPI-IS)}}
\date{NG SfL / 2016 March 30}

\newcommand{\conclusions}{%
\begin{frame}
  \frametitle{High-level summary}
  \begin{itemize}
  \item Exoplanet search involves extracting tiny, sparse signals, often \emph{smaller} than the dominant noise.
  \item Data-driven models of this ``noise'' outperform any theory, and also any simple filtering.
  \item General advice for next-generation projects:
    \begin{itemize}
    \item Take as much data as physically possible.
    \item Shorten exposure times!
    \item Build flexibility into observing strategy and details.
    \end{itemize}
  \end{itemize}
\end{frame}}

\begin{document}

\begin{frame}
  \titlepage
\end{frame}

\conclusions

\begin{frame}
  \frametitle{Hard problems}
  \begin{itemize}
  \item Like NG, we work on \emph{hard problems}.
    \begin{itemize}
    \item We work mainly in the software domain.
    \item When making a measurement looks impossible.
    \item (We recruit great people this way!)
    \end{itemize}
  \end{itemize}
\end{frame}

\begin{frame}
  \frametitle{Ideas from yesterday}
  \begin{itemize}
  \item Telescopes and instruments are \emph{integrated hardware--software systems}.
    \begin{itemize}
    \item Let's work out these trades.
    \item Let's involve software from day zero.
    \end{itemize}
  \item Sometimes your biggest challenges are your \emph{biggest opportunities}.
    \begin{itemize}
    \item example: \kepler\ \project{K2} pointing jitter vastly improved our instrument model.
    \item example: Extreme sparseness of transit signal permits sensitive stellar noise modeling.
    \end{itemize}
  \end{itemize}
\end{frame}

\begin{frame}
  \frametitle{\kepler\ reanalyses ({\footnotesize Foreman-Mackey \etal, 1406.3020})}
  ~\hfill
  \includegraphics<1>[height=\figureheight]{1406.3020/results-period.pdf}
  \includegraphics<2>[height=\figureheight]{1406.3020/results-radius.pdf}
  \includegraphics<3>[height=\figureheight]{1406.3020/results-rate.pdf}
  \includegraphics<4>[width=0.8\textwidth]{1406.3020/figures-comparison.pdf}
\end{frame}

\begin{frame}
  \frametitle{Self-calibration idea}
  \begin{itemize}
  \item If a star's behavior can be predicted confidently by
    other stars, then that behavior must be being imprinted by
    the spacecraft.
  \item We are using these ideas to separate spacecraft-induced from
    intrinsic stellar variability {\footnotesize (Wang \etal, arXiv:1508.01853;
    Sch\"olkopf \etal, arXiv:1505.03036)}.
  \end{itemize}
\end{frame}

\begin{frame}
  \frametitle{Data-driven idea}
  \begin{itemize}
  \item The best predictors of a star (or pixel's) behavior are the
    empirical behaviors of \emph{other stars} (or pixels).
  \item The data \emph{are} the model.
  \end{itemize}
\end{frame}

\begin{frame}
  \frametitle{Data-driven idea}
  \begin{itemize}
  \item Imagine that spacecraft pointing is the source of your
    dominant systematic.
  \item Imagine that you have light curves that are good to a part in
    $10^5$, every one of which depends differently on pointing.
  \item The light curves themselves contain the pointing information
    at very high fidelity.
  \item (Much higher fidelity than any measured centroids or s/c aspect solution!)
  \end{itemize}
\end{frame}

\begin{frame}
  \frametitle{\project{K2} reanalysis {\footnotesize (Foreman-Mackey \etal, 1502.04715)}}
  ~\hfill
  \includegraphics<1>[trim=100 100 100 100, clip, height=\figureheight]{brownbag/brownbagp10.pdf}
  \includegraphics<2>[trim=100 100 100 100, clip, height=\figureheight]{brownbag/brownbagp14.pdf}
  \includegraphics<3>[trim=100 100 100 100, clip, height=\figureheight]{brownbag/brownbagp15.pdf}
  \includegraphics<4>[trim=100 100 100 100, clip, height=\figureheight]{brownbag/brownbagp17.pdf}
  \includegraphics<5>[trim=100 100 100 100, clip, height=\figureheight]{brownbag/brownbagp18.pdf}
\end{frame}

\begin{frame}
  \frametitle{\project{K2} reanalysis {\footnotesize (Foreman-Mackey \etal, 1502.04715)}}
  ~\hfill
  \includegraphics<1>[height=\figureheight]{1502.04715/figures-periodic.pdf}
  \includegraphics<2>[height=\figureheight]{1502.04715/figures-de-trended.pdf}
  \includegraphics<3>[height=\figureheight]{1502.04715/figures-folded.pdf}
  \includegraphics<4>[height=\figureheight]{1502.04715/figures-candidates.pdf}
\end{frame}

\begin{frame}
  \frametitle{\project{P1640} spectroscopy {\footnotesize (Oppenheimer \etal, 1303.2627)}}
  ~\hfill
  \includegraphics<1>[height=\figureheight]{1303.2627/f8b.pdf}
\end{frame}

\begin{frame}
  \frametitle{\project{P1640} pipeline {\footnotesize (Fergus \etal, 1408.4248)}}
  ~\hfill
  \includegraphics<1>[width=\textwidth]{1408.4248/examples.pdf}
  \includegraphics<2>[height=\figureheight]{1408.4248/eigenvectors.pdf}
  \includegraphics<3>[height=\figureheight]{1408.4248/patch_reconstructions.pdf}
  \includegraphics<4>[width=\textwidth]{1408.4248/residuals.pdf}
  \includegraphics<5>[height=\figureheight]{1408.4248/mean_rt2.pdf}
\end{frame}

\begin{frame}
  \frametitle{Systematic noise model}
  \begin{itemize}
  \item Don't ``fit and subtract'' your systematics!
  \item That isn't conservative.
  \item Signals get distorted {\footnotesize (Foreman-Mackey \etal, 1502.04715)}
  \item You need to \emph{marginalize out the noise}.
  \end{itemize}
\end{frame}

\begin{frame}
  \frametitle{Earth-like transit signals}
  \begin{itemize}
  \item This requires good alignment (percent-level).
  \item Earth blocks $10^{-4}$ of the light from the Sun.
  \item It does this for 13 hours out of every 365.25 days.
  \item The Sun has stochastic variability with an \emph{amplitude larger than the signal}.
  \end{itemize}
\end{frame}

\begin{frame}
  \frametitle{Noise modeling}
  \begin{itemize}
  \item Build a very sophisticated, generative model of stellar variability.
    \begin{itemize}
    \item Different for every star.
    \item Capitalize on sparsity of the exoplanet signal!
    \end{itemize}
  \item Show that isolated transits (eclipses) cannot be generated by that model.
  \end{itemize}
\end{frame}

\begin{frame}
  \frametitle{\kepler\ reanalyses ({\footnotesize Foreman-Mackey \etal, in prep})}
  ~\hfill
  \includegraphics<1>[height=\figureheight]{dfm/full_sample.pdf}
\end{frame}

\begin{frame}
  \frametitle{Advice: Enormous models}
  \begin{itemize}
  \item Don't be afraid of enormous models.
    \begin{itemize}
    \item We sometimes use of order $10^9$ nuisance parameters for a
      single \kepler\ lightcurve (70,000 points) {\footnotesize (Wang
        \etal, arXiv:1508.01853)}.
    \end{itemize}
  \item In general, for nuisances, you want to use very flexible
    models but to learn the ``priors''\footnote{They are not really
      priors if you learn them!}.
  \item Or \emph{train and test} in disjoint sets.
  \end{itemize}
\end{frame}

\begin{frame}
  \frametitle{Advice: Data volume}
  \begin{itemize}
  \item More data is better.
  \item Push bandwidth limits.
  \item Shorten exposure times
    \begin{itemize}
    \item Astronomer folklore about read noise vs shot noise is \emph{often wrong}.
    \end{itemize}
  \end{itemize}
\end{frame}

\begin{frame}
  \frametitle{Advice: Vary everything}
  \begin{itemize}
  \item Vary pointing, exposure time as much as possible.
    \begin{itemize}
    \item We can show that \kepler\ would have returned more with variable exposure times.
    \item Exposure variations do not add to bandwidth.
    \end{itemize}
  \item Vary optics (focus, alignment) if you are permitted.
  \item Vary s/c control procedures (thruster firings, \etc).
  \item A star-shade adds six new degrees of freedom (at least) of variation!
  \end{itemize}
\end{frame}

\begin{frame}
  \frametitle{Advice: Physics-driven and data-driven teams}
  \begin{itemize}
  \item Task teams to take data-driven and theory-driven approaches.
  \item They will learn from each other.
  \end{itemize}
\end{frame}

\conclusions

\begin{frame}
~\hfill [backup slides] \hfill~
\end{frame}

\begin{frame}
  \frametitle{The paradox of precision astrophysics}
  \begin{itemize}
  \item models are incredibly \emph{explanatory}
    \begin{itemize}
    \item $\Lambda$CDM
    \item stellar spectroscopy
    \item helioseismology
    \end{itemize}
  \item and yet...
  \item<2-> models are \emph{wrong} (ruled out) in detail
    \begin{itemize}
    \item $\chi^2 \gg \nu$
    \item ``The $\chi^2$ statistic is a measure of the size of your data!''
    \end{itemize}
  \item<2-> missing physics, approximation, computation, \emph{gastrophysics}
  \end{itemize}
\end{frame}

\begin{frame}
  \frametitle{Data-driven models aren't fully interpretable}
  \begin{itemize}
  \item Use them for the parts of the problem you \emph{don't} care about.
  \end{itemize}
\end{frame}

\begin{frame}
  \frametitle{Generalizing (Gaussian) noise}
  \begin{itemize}
  \item $-2\,\ln L = \transpose{[y - m]}\cdot V^{-1}\cdot [y - m] + \ln\det V$
  \item Multiple addititive sources of (Gaussian) noise? Add covariances.
  \item Adding new components to the noise tensor $V$ is
    equivalent to fitting for new noise processes and marginalizing
    them out.
  \item This is literally as trivial as it sounds.
  \end{itemize}
\end{frame}

\end{document}
