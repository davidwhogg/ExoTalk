\documentclass[pdftex]{beamer}
%\input{vc}
% 1.77778 is the ratio of 16 to 9
\setlength{\paperheight}{3.5in}
\setlength{\paperwidth}{1.77778\paperheight}
% 1.33333 is the ratio of 4 to 3
%\setlength{\paperheight}{4.0in}
%\setlength{\paperwidth}{1.33333\paperheight}
\setlength{\textwidth}{0.85\paperwidth}
% import the next thing *after* the papersize
\input{./hogg_presentation} % hogg standard colors

\title{Extra-solar planets: Search, characterization, and population inferences}
\author[David W. Hogg (NYU)]{David W. Hogg \\
  \textsl{\small Center for Cosmology and Particle Physics,
                 New York University} \\
  \textsl{\small Center for Data Science,
                 New York University} \\
  \textsl{\small Max-Planck-Insitut f\"ur Astronomie, Heidelberg}}
\date{2015 May 13}

\newcommand{\conclusions}{%
\begin{frame}
  \frametitle{summary}
  \begin{itemize}
  \item exoplanet research is shaped by engineering and data analysis challenges
    \begin{itemize}
    \item I focus on these issues
    \end{itemize}
  \item \emph{search} involves extracting tiny, sparse signals from (huge) data sets
    \begin{itemize}
    \item noise modeling and marginalization of systematics
    \end{itemize}
  \item \emph{characterization} leaves us with noisy individual-planet measurements
  \item \emph{population inferences} require expensive noise propagation
    \begin{itemize}
    \item hierarchical probabilistic inference
    \end{itemize}
  \item Earth-like planets are plentiful in our Galaxy
    \begin{itemize}
    \item a percent (or more) of Sun-like stars host Earth-like planets
    \item the Solar System is not obviously typical in any way
    \item little is known about Jupiter-like planets
    \end{itemize}
  \end{itemize}
\end{frame}}

\begin{document}

\conclusions

\begin{frame}
  \titlepage
  in collaboration with:\\
  \emph{Dan~Foreman-Mackey}~(NYU),
  Ben~Montet~(Caltech),
  Tim~Morton~(Princeton),
  Bernhard~Sch\"olkopf~(MPI-IS),
  Dun~Wang~(NYU)
\end{frame}

\begin{frame}
  \frametitle{the NASA \kepler\ Mission}
  \begin{itemize}
  \item stared at 150,000 stars for 4 years (30-min cadence)
    \begin{itemize}
    \item 70,000 measurements per star
    \item all data completely public
    \end{itemize}
  \item looking for exoplanet transit signals
  \item found \emph{thousands} of exoplanets (candidates)
  \end{itemize}
\end{frame}

\begin{frame}
  \frametitle{the NASA \kepler\ Mission}
  ~\hfill
  \includegraphics<1>[height=\figureheight]{kepler/750603main_Ball_Kepler_A8468_275_lg_blog_main_horizontal.jpg}
  \includegraphics<2>[height=\figureheight]{kepler/Kepler_FOV_hiRes.jpg}
  \includegraphics<3>[height=\figureheight]{kepler/FirstLightLogInvertedPink_wslbld2400.jpg}
  \includegraphics<4>[height=0.9\figureheight]{1502.04715/figures-de-trended.pdf}
  \includegraphics<5>[height=0.9\figureheight]{1502.04715/figures-folded.pdf}
\end{frame}

\begin{frame}
  \frametitle{the NASA \kepler\ K2 Mission}
  \begin{itemize}
  \item failure of reaction wheels led to worse pointing
    \begin{itemize}
    \item forced to stare at fields such that Solar torque is near zero
    \end{itemize}
  \item staring at 12 fields for 80 days each
  \item all data immediately public
  \item finding exoplanets around lower-mass stars
    \begin{itemize}
    \item currently the longest list of K2 discoveries is from my group
    \item Foreman-Mackey \etal, arXiv:1502.04715
    \end{itemize}
  \end{itemize}
\end{frame}

\begin{frame}
  \frametitle{an Earth-like transit signal}
  \begin{itemize}
  \item requires good alignment (percent-level)
  \item Earth blocks $10^{-4}$ of the light from the Sun
  \item it does this for 13 hours out of every 365.25 days
  \item the Sun has stochastic variability with an amplitude larger than the signal
  \end{itemize}
\end{frame}

\begin{frame}
  \frametitle{noise models}
  \begin{itemize}
  \item precision astrophysics is all about noise
  \item stochastic variability of the Sun
    \begin{itemize}
    \item models of the Sun do not predict variability in detail
    \item we use data-driven models (non-parametrics)
    \item auto-regressive models, Gaussian Processes
    \end{itemize}
  \item spacecraft issues
    \begin{itemize}
    \item telescope pointing is not precise (especially for K2)
    \item point-spread function and flat-field not known to sufficient precision
    \item temperature changes
    \item we make use of shared information across all stars
    \end{itemize}
  \item all of these noise sources are \emph{larger than the signals we care about}
  \end{itemize}
\end{frame}

\begin{frame}
  \frametitle{causal idea}
  \begin{itemize}
  \item if a star's behavior can be predicted confidently by
    \emph{other stars} then that behavior must be being imprinted by
    the spacecraft
  \item using these ideas to separate spacecraft-induced from
    intrinsic stellar variability (Wang \etal, in preparation)
  \end{itemize}
\end{frame}

\begin{frame}
  \frametitle{noise modeling {\footnotesize (Foreman-Mackey \etal, arXiv:1502.04715)}}
  ~\hfill
  \includegraphics<1>[trim=100 100 100 100, clip, height=\figureheight]{brownbag/brownbagp10.pdf}
  \includegraphics<2>[trim=100 100 100 100, clip, height=\figureheight]{brownbag/brownbagp11.pdf}
  \includegraphics<3>[trim=100 100 100 100, clip, height=\figureheight]{brownbag/brownbagp14.pdf}
  \includegraphics<4>[trim=100 100 100 100, clip, height=\figureheight]{brownbag/brownbagp15.pdf}
  \includegraphics<5>[trim=100 100 100 100, clip, height=\figureheight]{brownbag/brownbagp17.pdf}
\end{frame}

\begin{frame}
  \frametitle{exoplanet search}
  \begin{itemize}
  \item enormous set of hypothesis tests
    \begin{itemize}
    \item extremely fine grid in period and phase
    \item can't ``just'' use Fourier methods because of signal sparsity
    \end{itemize}
  \item very flexible noise model
    \begin{itemize}
    \item effectively thousands to millions of parameters per lightcurve
    \item fully marginalize out the noise draw
    \end{itemize}
  \item clever applied mathematics
    \begin{itemize}
    \item very fast linear algebra for Gaussian Processes
    \item Ambikasaran \etal, arXiv:1403.6015
    \item approximations that permit re-use of repeated calculation
    \item exploit embarassing parallelization
    \end{itemize}
  \end{itemize}
\end{frame}

\begin{frame}
  \frametitle{marginalizing out noise}
  \begin{itemize}
  \item $\displaystyle\chi^2\equiv\transpose{[y - \mu]}\cdot C^{-1}\cdot[y - \mu]$
  \item $\displaystyle C = \diag(\sigma^2)$
  \item<2-> $\displaystyle C = \diag(\sigma^2) + B\cdot\Lambda\cdot\transpose{B}$
    \begin{itemize}
    \item $B$ contains the $M=150$ (say) basis vectors
    \item either $\Lambda\rightarrow\infty$ (permitted!) or choose $\Lambda$ sensibly
    \item marginalizes (projects) out anything in the $B$ subspace
    \end{itemize}
  \item<3-> $\displaystyle C = \diag(\sigma^2) + B\cdot\Lambda\cdot\transpose{B} + K$
    \begin{itemize}
    \item $\displaystyle K_{nn'} = k(|t_{n} - t_{n'}|)$
    \item for example, $\displaystyle K_{nn'} = a\,\exp -\frac{1}{2}\,\frac{[t_{n} - t_{n'}]^2}{\tau^2}$
    \item marginalizes out smooth, nonlinear functions of time (a Gaussian Process)
    \end{itemize}
  \item<4-> These methods marginalize out (Gaussian) non-trivial noise models.
    \begin{itemize}
    \item equivalent to integrating over all noise draws consistent with the data
    \item there are more extreme models that permit nonlinear functions of the $B$ vectors
    \end{itemize}
  \end{itemize}
\end{frame}

\begin{frame}
  \frametitle{exoplanet search {\footnotesize (Foreman-Mackey \etal, arXiv:1502.04715)}}
  ~\hfill
  \includegraphics<1>[height=0.9\figureheight]{1502.04715/figures-corr.pdf}
  \includegraphics<2>[height=0.9\figureheight]{1502.04715/figures-de-trended.pdf}
  \includegraphics<3>[height=0.9\figureheight]{1502.04715/figures-linear.pdf}
  \includegraphics<4>[height=0.9\figureheight]{1502.04715/figures-periodic.pdf}
  \includegraphics<5>[height=0.9\figureheight]{1502.04715/figures-de-trended.pdf}
  \includegraphics<6>[height=0.9\figureheight]{1502.04715/figures-folded.pdf}
  \includegraphics<7>[height=\figureheight]{1502.04715/figures-candidates.pdf}
\end{frame}

\begin{frame}
  \frametitle{false positives and false negatives}
  \begin{itemize}
  \item it is easy to figure out the \emph{false-negative} rate
    \begin{itemize}
    \item inject fake signals into the lightcurves, search
    \item complement is called the ``efficiency'' or ``completeness''
    \end{itemize}
  \item<2-> it is \emph{impossible} to figure out the \emph{false-positive} rate
    \begin{itemize}
    \item ground truth is known for no complete sample
    \item need a model for all sources (instrumental and astrophysical) of false positive signals
    \item the most interesting objects are nearly impossible to follow up
    \end{itemize}
  \item<3-> \emph{secret fact:}
    it looks like many of the most interesting exoplanet candidates may be false positives
    \begin{itemize}
    \item long periods, small radii
    \end{itemize}
  \end{itemize}
\end{frame}

\begin{frame}
  \frametitle{search completeness {\footnotesize (Foreman-Mackey \etal, arXiv:1502.04715)}}
  ~\hfill
  \includegraphics<1>[height=\figureheight]{1502.04715/figures-completeness.pdf}
\end{frame}

\begin{frame}
  \frametitle{characterization: what can we measure?}
  \begin{itemize}
  \item easy: period, size ratio
  \item hard: size, temperature, composition
    \begin{itemize}
    \item We can't understand the exoplanets any better than we can understand the stars they orbit!
    \item renewed interest in precision stellar astrophysics
    \end{itemize}
  \end{itemize}
\end{frame}

\begin{frame}
  \frametitle{how do you use noisy measurements?}
  \begin{itemize}
  \item exoplanetary astronomers tend to think Bayesian
    \begin{itemize}
    \item provide posterior information about exoplanet properties
    \end{itemize}
  \item we want to know about the \emph{population} of exoplanets
    \begin{itemize}
    \item exoplanets are ``samples'' from this population
    \item observations are distorted by samples from the noise process
    \end{itemize}
  \item usually think in terms of generative probabilistic model
    \begin{itemize}
    \item population parameters characterize the population model
    \item population model generates exoplanet parameters
    \item exoplanet parameters characterize one exoplanetary system model
    \item exoplanetary system model (and noise model) generates the raw data
    \end{itemize}
  \item want to make \emph{noise-marginalized population inferences}
  \end{itemize}
\end{frame}

\begin{frame}
  \frametitle{hierarchical probabilistic modeling}
  \begin{itemize}
  \item if an observer gives us posterior information, it includes some \emph{prior}
  \item our population model is a \emph{replacement} for that prior
    \begin{itemize}
    \item so we want likelihood functions, not posterior samples
    \item (likelihood \emph{functions} not maximum-likelihood parameters)
    \end{itemize}
  \item more applied math: importance sampling tricks
    \begin{itemize}
    \item can re-weight posterior samples to use them like likelihood information
    \end{itemize}
  \item the full model contains a lot of complicated detail
    \begin{itemize}
    \item very flexible populations model with smoothness prior
    \item bespoke elliptical slice sampling method for inference
    \item original \kepler\ Mission data but independent Petigura \etal\ catalog and false-negative estimates
    \end{itemize}
  \end{itemize}
\end{frame}

\begin{frame}
  \frametitle{importance sampling}
  \begin{itemize}
  \item you can only combine observations of different exoplanets if you have \emph{likeihoods}
  \item in reality you will have \emph{posteriors}
    \begin{itemize}
    \item for both bad reasons (laziness) good reasons (use of data-driven models)
    \item tend to be made with a crazy prior $I_0$
    \end{itemize}
  \item replace the crazy prior $I_0$ with a population model $\alpha$ by
    \begin{eqnarray}
      p(y\given \alpha) &=& \frac{1}{Z}\,\int \frac{p(\theta\given y,\alpha)}{p(\theta\given y,I_0)}\,p(\theta\given y,I_0)\,\dd\theta
      \\
      &\approx& \frac{1}{Z}\sum_k \frac{p(\theta_k\given y,\alpha)}{p(\theta_k\given y,I_0)}
    \end{eqnarray}
  \end{itemize}
\end{frame}

\begin{frame}
  \frametitle{Exoplanet populations {\footnotesize (Foreman-Mackey \etal, arXiv:1406.3020)}}
  ~\hfill
  \includegraphics<1>[height=\figureheight]{1406.3020/results-results.pdf}
  \includegraphics<2>[height=\figureheight]{1406.3020/results-period.pdf}
  \includegraphics<3>[height=\figureheight]{1406.3020/results-radius.pdf}
  \includegraphics<4>[height=\figureheight]{1406.3020/results-linear-radius.pdf}
  \includegraphics<5>[height=\figureheight]{1406.3020/results-rate.pdf}
  \includegraphics<6>[width=0.8\textwidth]{1406.3020/figures-comparison.pdf}
\end{frame}

\conclusions

\end{document}
